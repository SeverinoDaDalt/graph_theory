\documentclass[12pt]{article}
\usepackage[margin=1in]{geometry}
\usepackage{xcolor}
\usepackage{amsmath,amsthm,amssymb,scrextend}
\usepackage{mathtools}
\usepackage{fancyhdr}
\usepackage{todonotes}
\pagestyle{fancy}

\newcommand{\cont}{\subseteq}
\usepackage{tikz}
\usepackage{pgfplots}
\usepackage[mathscr]{euscript}
\let\euscr\mathscr \let\mathscr\relax% just so we can load this and rsfs
\usepackage[scr]{rsfso}
\usepackage{multicol}
\usepackage[colorlinks=true, pdfstartview=FitV, linkcolor=blue,
citecolor=blue, urlcolor=blue]{hyperref}
<<<<<<<< HEAD:main_5.2.tex

% My packages:
\usepackage{bbm}

========
>>>>>>>> origin/tmp:exercises_1.tex
\DeclareMathOperator{\arcsec}{arcsec}
\DeclareMathOperator{\arccot}{arccot}
\DeclareMathOperator{\arccsc}{arccsc}
\newcommand{\ddx}{\frac{d}{dx}}
\newcommand{\dfdx}{\frac{df}{dx}}
\newcommand{\ddxp}[1]{\frac{d}{dx}\left( #1 \right)}
\newcommand{\dydx}{\frac{dy}{dx}}
\let\ds\displaystyle
\newcommand{\intx}[1]{\int #1 \, dx}
\newcommand{\intt}[1]{\int #1 \, dt}
\newcommand{\defint}[3]{\int_{#1}^{#2} #3 \, dx}
\newcommand{\imp}{\Rightarrow}
\newcommand{\un}{\cup}
\newcommand{\inter}{\cap}
\newcommand{\ps}{\mathscr{P}}
\newcommand{\set}[1]{\left\{ #1 \right\}}
\newtheorem*{claim}{Claim}
<<<<<<<< HEAD:main_5.2.tex
\newtheorem*{remark}{Remark}
\newtheorem*{problem}{Problem 2}
========
\newtheorem{problem}{Problem}
\theoremstyle{definition}
\newtheorem*{sol}{Solution}

>>>>>>>> origin/tmp:exercises_1.tex
\usepackage{enumitem}
\usepackage{color}
\usepackage{amsfonts}
\usepackage{stmaryrd}
\renewcommand{\labelenumii}{\theproblem.\arabic{enumi}.\arabic{enumii}}
\begin{document}
<<<<<<<< HEAD:main_5.2.tex
\lhead{Graph Theory. Exercise 2 from Set 5.}
\rhead{Severino Da Dalt}
% PROBLEM 2
\begin{problem}
    Show that if $p \geq 5 \sqrt{\frac{\log{n}}{n}}$ then a.a.s. every edge of $\mathbb{G}_{n,p}$ is contained in at least $\frac{n p^2}{2}$ triangles.
    \begin{sol}
        \begin{proof}
            First of all, let's bound the probability of an edge $e$ of $\mathbb{G}_{n,p}$ being contained in less than $\frac{n p^2}{2}$ triangles.
            Take $u$ and $v \in V(\mathbb{G}_{n,p})$ the vertices connected by the edge $e$.
            For all $w_i \in V(\mathbb{G}_{n,p}) \setminus \{u, v\}$ consider the random variable
            \[ Y_i = \mathbbm{1}(\{w_i u, w_i v\} \subseteq E(\mathbb{G}_{n,p})) \sim Ber(p^2) \]
            Since the edges $w_i u, w_i v, w_j u, w_j v$ are independent for all $i \neq j$, $(Y_i)_i$ is a collection of independent random variables.
            Thus,
            \[ Y = \sum_{i=1}^{n-2} Y_i \sim Bin(n-2, p^2) \]

            \begin{claim}
                $\frac{n p^2}{2} = \frac{n}{2(n-2)} (n-2) p^2 = (1 - (1 - \frac{n}{2(n-2)})) (n-2) p^2$
            \end{claim}

            \paragraph
            \noindent Now, notice that $e$ being contained in a triangle in $\mathbb{G}_{n,p}$ is equivalent to the existence of a vertex $v_i$ connected to both $u$ and $v$ by an edge (i.e. $Y_i = 1$).
            Thus, we can calculate the probability of the number of triangles containing $e$ being smaller than $\frac{n p^2}{2}$ as:
            \[ \mathbb{P}(Y < \frac{n p^2}{2}) \leq \mathbb{P}(Y \leq \frac{n p^2}{2}) = \mathbb{P}(Y \leq (1 - (1 - \frac{n}{2(n-2)})) (n-2) p^2) = \mathbb{P}(Y \leq (1 - \delta) \mu) \]
            where $\mu \coloneqq \mathbb{E}(Y) = (n-2) p^2$ and $\delta \coloneqq 1 - \frac{n}{2 (n-2)} < 1$
            Then, we can use Chernoff's Inequality (Theorem 2.1 of session 5):
            \[ \mathbb{P}(Y \leq (1 - \delta) \mu) \leq e^{-\delta^2 \mu / 2} \]
            Finally, we can calculate (bound) the probability of the existence of an edge not satisfying the condition:
            \[ \mathbb{P}(\exists e, Y < \frac{np^2}{2}) \leq \binom{n}{2} e^{-\delta^2 \mu / 2} \leq \binom{n}{2} e^{- (1 - \frac{n}{2(n-2)})^2 (n-2) p^2 / 2} \]
            Now, since $p \geq 5 \sqrt{\frac{\log{n}}{n}}$ we can rewrite as:
            \[ p = 5 f \sqrt{\frac{\log{n}}{n}} \text{ with } f \geq 1 \]
            Substitute $p$ in the formula:
            \[ \mathbb{P}(\exists e, Y < \frac{np^2}{2}) \leq \binom{n}{2} e^{- (1 - \frac{n}{2(n-2)})^2 (n-2) (5 f)^2 \frac{\log{n}}{n} / 2} \leq \]
            \[ \leq \binom{n}{2} (e^{\log{n}})^{- (1 - \frac{n}{2(n-2)})^2 \frac{n-2}{n} (5 f)^2 / 2} \leq \frac{n (n-1)}{2 n^{(1 - \frac{n}{2 (n-2)})^2 \frac{n-2}{n} (5 f)^2 / 2}} \]
            Since $(1 - \frac{n}{2 (n-2)})^2 \frac{n-2}{n} (5 f)^2 / 2 \rightarrow \frac{25}{8} f^2 > 2$ when $n \rightarrow \infty$,
            \[ \mathbb{P}(\exists e, Y < \frac{np^2}{2}) \rightarrow 0 \]
            proving the statement of the problem.

            \begin{remark}
                Notice that the statement still holds if we lower the bound of $p$ to $p > 4\sqrt{\frac{\log{n}}{n}}$.
            \end{remark}
        \end{proof}
    \end{sol}
\end{problem}

\end{document}
========
\lhead{Communative Algebra. Exercises from Set 1.}
\rhead{Ferran Espuña, Severino Da Dalt}
\input{exercises_1/ex_1}
\input{exercises_1/ex_2}
\input{exercises_1/ex_3}
\input{exercises_1/ex_4}
\input{exercises_1/ex_5}
\input{exercises_1/ex_6}
\input{exercises_1/ex_7}
\input{exercises_1/ex_8}
\input{exercises_1/ex_9}
\input{exercises_1/ex_10}
\input{exercises_1/ex_11}
\input{exercises_1/ex_12}
\input{exercises_1/ex_13}
\end{document}
>>>>>>>> origin/tmp:exercises_1.tex
